\documentclass[12pt, notitlepage]{article}   	% use "amsart" instead of "article" for AMSLaTeX format
\usepackage{geometry}                		% See geometry.pdf to learn the layout options. There are lots.
\geometry{a4paper}                   		% ... or a4paper or a5paper or ... 
%\geometry{landscape}                		% Activate for rotated page geometry
\usepackage[parfill]{parskip}    		% Activate to begin paragraphs with an empty line rather than an indent
\usepackage{graphicx}				% Use pdf, png, jpg, or eps§ with pdflatex; use eps in DVI mode
								% TeX will automatically convert eps --> pdf in pdflatex

\usepackage{hyperref}
		
%SetFonts

\usepackage[T1]{fontenc}
\usepackage[utf8]{inputenc}

\usepackage{tgbonum}

%SetFonts

\title{
	\textbf{
		BIOL 6100-029
	} \\
	\large Adv. Topics in Biology: Plant physiological theory and techniques \\
	\large Spring 2023
}

\date{\vspace{-5ex}}

\begin{document}

{\fontfamily{phv}\selectfont %select helvetica (code = phv)

\maketitle

\section{Course Description}
Students in this course will learn mechanistic theory and techniques commonly used in
the field of plant physiology. The course will focus on data acquisition at scales from
the organ to the whole-plant. Theory will be taught through reading of primary literature,
which will be combined with the teaching of techniques in the lab and field.

\subsection{Class Time and Location}
Fridays 10:00-10:50 or otherwise agreed upon

Experimental Sciences Building II (ESB II) Room 409 or 
or otherwise agreed upon.

\subsection{Instructor}
Dr. Nick Smith \par
ESB II Room 402D \par
806-834-7363 \par
nick.smith@ttu.edu \par
\textit{Meetings by appointment}

\subsection{Recommended Texts}
Plant Physiological Ecology (2nd Edition; 2008) by Lambers, Chapin, and Pons \par
The book can be accessed from Springer here: 
\url{https://www.springer.com/us/book/9780387783406}. Click on "Access this title on 
SpringerLink." It can also be accessed through the TTU library. \par
Plant Physiology and Development (6th Edition) by Taiz, Ziegler, Moller, and Murphy

\section{Mode of Instruction}
All instruction will be done face-to-face unless the university directs classes be 
taught online.

\section{Course Materials}
All course materials, including lecture slides, readings, activities, and code 
will be posted to a GitHub repository for the course.
The primary repository address is
\url{https://github.com/SmithEcophysLab/biol6100_spring2023}.

\section{Learning Objective}
This course will broadly focus on understanding the theory behind techniques
used in plant physiology, with a focus on those use in plant physiological ecology. 
The course will also use hands-on lessons to teach students how to perform those
techniques.
Class activities will be based on discussion and dissemination of ideas, 
including classic and recent scientific literature. 
Topics will be flexible and modified to match student interests where possible.

\section{Attendance Policy}
Attendance is strongly recommended. 
The course assessments will be done during class (see below).

\section{Course Assessment}
\subsection{\textit{Participation and Engagement}}
Being an active and engaged participant in the class will benefit your understanding
of material as well as your peers'. Examples include asking questions, providing feedback,
and facilitating discussion. Participation and engagement of each student will be monitored
during each class period and will constitute the only assessment.

\section{Grading}
Participation and Engagement: 100\% \par

\section{Grading Scale}
A: $\geq$ 90\% \par
B: 80 – 90\% \par
C: 70 – 80\% \par
D: 60 – 70\% \par
F: $\leq$ 59.9\% \par

\section{Missing In-class Activities}
Students will be required to be in class to receive participation and engagement points. 
Please read below if class is to be missed due to an officially approved trip, illness, or special circumstance:

\subsection{Officially Approved Trips}
The person responsible for a student representing the University on officially approved 
trips should notify the instructor of the departure and return schedules in advance.  
The student will not be penalized for the absence but is responsible for the material missed.

\subsection{Illness Based Absence Policy}
If at any time during this semester you feel ill, in the interest of your own health and 
safety as well as the health and safety of your instructors and classmates, you are 
encouraged not to attend face-to-face class meetings or events.  Please review the steps 
outlined below that you should follow to ensure your absence for illness will be excused:
\begin{itemize}
	\item{Call Student Health Services at 806.743.2848 or your health care provider.}
	\item{Contact Dr. Smith to let him know of your situation.}
	\item{If appropriate, obtain and return a doctor's note to Dr. Smith.}
\end{itemize}

\subsection{Special Circumstance Absence}
There may be special circumstances that render missing class unavoidable.
If this arises, please let Dr. Smith know of the situation as soon as possible,
so that the loss of point due to the absence can be discussed.

\section{COVID-19 Statement}
The University will continue to monitor CDC, State, and TTU System guidelines concerning COVID-19. 
Any changes affecting class policies or temporary changes to delivery modality will be in 
accordance with those guidelines and announced as soon as possible. Students will not be 
required to purchase specialized technology to support a temporary course modality change, 
though students are expected to have access to a computer to access course content and 
course-specific messaging as needed. 

If you test positive for COVID-19, report your positive test through TTU's reporting system: 
\url{https://www.depts.ttu.edu/communications/emergency/coronavirus/}. Once you report a positive 
test, the portal will automatically generate a letter that you can distribute to your 
professors and instructors.

The TTU COVD-19 resource page is here: \url{https://www.depts.ttu.edu/communications/emergency/coronavirus/}.

\section{Special Considerations}
\subsection{ADA Statement}
Any student who, because of a disability, may require special arrangements in order to 
meet the course requirements should contact the instructor as soon as possible to make 
any necessary arrangements. Students should present appropriate verification from Student 
Disability Services during the instructor's office hours. Please note: instructors are 
not allowed to provide classroom accommodations to a student until appropriate verification 
from Student Disability Services has been provided. For additional information, please 
contact Student Disability Services in West Hall or call 806-742-2405.

\subsection{Religious Holy Days}
“Religious holy day” means a holy day observed by a religion whose places of worship 
are exempt from property taxation under Texas Tax Code §11.20. 
A student who intends to observe a religious holy day should make that intention known 
in writing to the instructor prior to the absence. 
A student who is absent from classes for the observance of a religious holy day shall be 
allowed to take an examination or complete an assignment scheduled for that day within a 
reasonable time after the absence. 
A student who is excused may not be penalized for the absence; however, the instructor 
may respond appropriately if the student fails to complete the assignment satisfactorily.

\section{TTU Resources for Discrimination, Harassment, and Sexual Violence}
Texas Tech University is committed to providing and strengthening an educational, 
working, and living environment where students, faculty, staff, and visitors are 
free from gender and/or sex discrimination of any kind. 
Sexual assault, discrimination, harassment, and other Title IX violations are 
not tolerated by the University. 
Report any incidents to the Office for Student Rights & Resolution, 
(806)-742-SAFE (7233), or file a report online at \url{https://www.depts.ttu.edu/titleix/}. 
Faculty and staff members at TTU are committed to connecting you to resources on campus. 
Some of these available resources are: TTU Student Counseling Center, 806-742-3674, 
\url{https://www.depts.ttu.edu/scc/} (provides confidential support on campus). 
TTU Student Counseling Center 24-hour Helpline, 806-742-5555, 
(assists students who are experiencing a mental health or interpersonal violence crisis; 
if you call the helpline, you will speak with a mental health counselor). 
Voice of Hope Lubbock Rape Crisis Center, 806-763-7273, \url{voiceofhopelubbock.org} 
(24-hour hotline that provides support for survivors of sexual violence). 
The Risk, Intervention, Safety and Education (RISE) Office, 806-742-2110, 
\url{https://www.depts.ttu.edu/rise/} (provides a range of resources and 
support options focused on prevention education and student wellness). 
Texas Tech Police Department, 806-742-3931, \url{http://www.depts.ttu.edu/ttpd/} 
(to report criminal activity that occurs on or near Texas Tech campus). 

\section{LGBTQIA}
Please contact the Office of LGBTQIA, Student Union Building Room 201, 806-742-5433, 
\url{www.lgbtqia.ttu.edu}. 
Within the Center for Campus Life, the Office serves the Texas Tech community 
through facilitation and leadership of programming and advocacy efforts. 
This work is aimed at strengthening the lesbian, gay, bisexual, transgender, queer, 
intersex, and asexual (LGBTQIA) community and sustaining an inclusive campus that 
welcomes people of all sexual orientations, gender identities, and gender expressions. 

\section{Classroom Civility}
Texas Tech University is a community of faculty, students, and staff that enjoys 
an expectation of cooperation, professionalism, and civility during the conduct of all 
forms of university business, including the conduct of student–student and student–faculty 
interactions in and out of the classroom. 
Further, the classroom is a setting in which an exchange of ideas and creative thinking 
should be encouraged and where intellectual growth and development are fostered. 
Students who disrupt this classroom mission by rude, sarcastic, threatening, abusive or 
obscene language and/or behavior will be subject to appropriate sanctions according to 
university policy.  Likewise, faculty members are expected to maintain the 
highest standards of professionalism in all interactions with all constituents of the 
university (\url{www.depts.ttu.edu/ethics/matadorchallenge/ethicalprinciples.php}).

\section{Academic Integrity}
Academic integrity is taking responsibility for one’s own class and/or course work, 
being individually accountable, and demonstrating intellectual honesty and ethical behavior. 
Academic integrity is a personal choice to abide by the standards of intellectual honesty 
and responsibility. 
Because education is a shared effort to achieve learning through the exchange of ideas, 
students, faculty, and staff have the collective responsibility to build mutual trust and respect. 
Ethical behavior and independent thought are essential for the highest level of academic 
achievement, which then must be measured. 
Academic achievement includes scholarship, teaching, and learning, all of which are shared endeavors. 
Grades are a device used to quantify the successful accumulation of knowledge through learning. 
Adhering to the standards of academic integrity ensures grades are earned honestly. 
Academic integrity is the foundation upon which students, faculty, and staff build their 
educational and professional careers. [Reference: Texas Tech University Quality 
Enhancement Plan, Academic Integrity Task Force, 2010].

\section{Plagiarism Statement}
Texas Tech University expects students to “understand the principles of academic integrity 
and abide by them in all class and/or course work at the University” (OP 34.12.5). 
Plagiarism is a form of academic misconduct that involves (1) the representation of words, 
ideas, illustrations, structure, computer code, other expression, or media of another as 
one's own and/or failing to properly cite direct, paraphrased, or summarized materials; 
or (2) self-plagiarism, which involves the submission of the same academic work more than 
once without the prior permission of the instructor and/or failure to correctly cite 
previous work written by the same student. Please review Section B of the TTU 
Student Handbook for more information related to other forms of academic misconduct, 
and contact your instructor if you have questions about plagiarism or other 
academic concerns in your courses. To learn more about the importance of 
academic integrity and practical tips for avoiding plagiarism, explore the 
resources provided by the TTU Library and the School of Law.

\section{Statement about Food Insecurity}
Any student who faces challenges securing their food or housing and believes this may 
affect their performance in the course is urged to contact the Dean of Students for 
support. Furthermore, please notify the professor if you are comfortable in doing so. 
The TTU Food Pantry is in Doak Hall 117. Please visit the website for hours of 
operation at \url{https://www.depts.ttu.edu/dos/foodpantry.php}.

\section{Creating Livable Futures}
This class is part of a campus-wide initiative called Creating Livable Futures, 
which is sponsored in part by the Texas Tech Center for Global Communication. 
As such, one of our objectives is to prepare you to communicate, 
in a fully interdisciplinary and global way, the challenges posed by pressing issues 
that speak to our collective wellbeing and sustainability. You will be asked to translate 
and communicate the work of leading thinkers on sustainability, and to expand discussing 
those materials through research experience and experiential learning.
These objectives will be met through discussion leads and the review paper. 

Your progress in communicating about global issues will be evaluated according to the 
Center for Global Communication rubric, so you will be invited to participate 
in one or more Creating Livable Futures activities outside of class that will 
complement class content. 
Planned Creating Livable Futures activities include participating in and attending 
speaker events and conferences, edit-a-thons, blogging and publication opportunities, 
student organizations, a book club, and even small scholarship opportunities for research. 

You’ll be informed of relevant opportunities and activities as they arise over 
the course of the semester.

\newpage

\section*{Schedule of Topics by Week}
09/01/23 - Introductions, semester planning, and goals \par
16/01/23 – Why plant physiology? \par
23/01/23 – Photosynthesis I: basics \par
30/01/23 – Photosynthesis I lab \par
06/02/23 – NO CLASS \par
13/02/23 – NO CLASS \par
20/02/23 – Photosynthesis II: response curves \par
27/02/23 – Photosynthesis II lab \par
06/03/23 – NO CLASS \par
13/03/23 – NO CLASS \par
20/03/23 – Fluorescence/conductance \par
27/03/23 – Fluorescence/conductance lab \par
03/04/23 – NO CLASS \par
10/04/23 – NO CLASS \par
17/04/23 – Tissue chemistry \par
24/04/23 – NO CLASS \par
01/05/23 – Tissue chemistry lab \par

} %end font selection

\end{document} 
